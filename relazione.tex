\documentclass{article}


\usepackage[italian]{babel}
\usepackage[utf8]{inputenc}


\usepackage{listings}



\begin{document}
\begin{titlepage}
    \begin{center}
        \Huge{Progetto Basi di Dati}\\
        [20mm]
        \Large{Paolo Addis}\\
        \Large{Samuele Poz}\\
        \Large{Tristano Munini}\\
        [20mm]
        \Large{RENDERE CARINO}
    \end{center}
\end{titlepage}

\tableofcontents
\thispagestyle{empty}
\cleardoublepage
\setcounter{page}{1}


% TODO link da indice a sections


\section{Progettazione Concettuale}\label{sec:prog_conc}

\subsection{Raccolta ed Analisi dei Requisiti}

Si vuole modellare il seguente insieme di informazioni riguardanti un sistema per la gestione delle diagnosi e delle terapie dei pazienti ricoverati in un dato ospedale.
\begin{itemize}
  \item Di ogni ricovero, il sistema deve memorizzare il codice univoco, il nome della divisione ospedaliera (Cardiologia, Reumatologia, Ortopedia, \dots ), il paziente ricoverato, le date di inizio e fine del ricovero e il motivo principale del ricovero.
  \item Di ogni paziente, il sistema deve memorizzare il codice sanitario (univoco), il cognome, il nome, la data di nascita, il luogo di nascita e la provincia di residenza.
    Per i pazienti residenti fuori regione, vengono memorizzati anche il nome della ULSS e la regione di appartenenza.
  \item Inoltre di ogni paziente si vuole memorizzare la cartella clinica, caratterizzata dall'insieme di tutte le diagnosi del paziente.
  \item Di ogni diagnosi effettuata durante il ricovero del paziente, sono memorizzati la patologia diagnosticata, col suo codice ICD10 (classificazione internazionale delle patologie) e l’indicazione della sua gravità (grave: si/no), la data e il nome e cognome del medico che ha effettuato la diagnosi.
  \item Nella base di dati si tiene traccia delle terapie prescritte ai pazienti durante il ricovero.
    Di ogni terapia, si memorizzano il farmaco prescritto, la dose giornaliera, le date di inizio e di fine della prescrizione, la modalità di somministrazione ed il medico che ha prescritto la terapia.
    % manca parte in cui si dice che la terapia è un concetto astratto
  \item Di ogni farmaco sono memorizzati il nome commerciale (univoco), l’azienda produttrice, il nome e la quantità dei principi attivi contenuti e la dose giornaliera raccomandata.
  \item Si tiene, infine, traccia delle diagnosi che hanno motivato le terapie.
    In particolare, ogni terapia è prescritta al fine di curare una o più patologie diagnosticate.
    Può capitare anche che una nuova patologia (registrata come nuova diagnosi) sia causata, come effetto collaterale, da una terapia precedentemente prescritta.
    Tale legame causa-effetto va registrato nella base di dati.
\end{itemize}
Tramite un portale web dovrà essere possibile accedere a tale base di dati ed aggiungere o rimuovere pazienti, ricoveri, diagnosi con relative terapie e nuovi farmaci. 

\subsection{Progettazione del Modello E-R}
TODO
\begin{itemize}
  \item Descrizione procedimento Inside-Out con schemi parziali (fasi della costruzione) e descrizione testuale delle relazioni e entità
    \begin{itemize}
      \item Unione in attributo composto e multivalore dei principi attivi 
    \end{itemize}
\end{itemize}

\clearpage
\section{Progettazione Logica}
TODO
\begin{itemize}
  \item Semplificazione dei concetti
    \begin{itemize}
      \item Rimozione della relazione ternaria (vedi schema "causa effetto")
    \end{itemize}
  \item Analisi delle ridondanze (vedi fogli "cartella clinica")
  \item Rimozione delle Generalizzazioni (non presente)

  \item Partizione/merging di entità
    \begin{itemize}
      \item Eliminazione dell'attributo composto "patologia" in DIAGNOSI
      \item Partizionamento dei principi attivi in una nuova entità
      \item Cose che si potrebbero aggiungere:
        \begin{itemize}
          \item Partizionamento entità inteso come separazione degli attributi in base alle operazioni (almeno le principali)
          \item Partizionamento di una relazione (si può fare con MEDICO collegato a PAZIENTE e DIAGNOSI) ???????
            (secondo Tri 'fare solo se non abbiamo abbastanza materiale')
        \end{itemize}
    \end{itemize}
  \item Selezione delle chiavi
    \begin{itemize}
      \item Aggiungere un codice come chiave della diagnosi (da giustificare)
    \end{itemize}
\end{itemize}
TODO nei commenti
\subsection{Ristrutturazione del Modello E-R}
\subsection{Traduzione del Modello Logico}

% SCRIVIAMO LA LISTA DELLE TABELLE CON TUTTE LE CHIAVI BELLE DENTRO 
% SOTTO SCRIVIAMO LA LISTA: "LA RELAZIONE X è GARANTITA DA QUESTA COMBO 
%                            DI CHIAVI (Y,Z)..."
% 
% - Traduzione di VIENE-RICOVERATO (da scrivere bene perchè è la prima)
% 
% PAZIENTE (_CF_, NOME,...)
% RICOVERO (_CODICE-RICOVERO_,...,DIVISIONE-OSPEDALIERA, CF, MOTIVO)
%           . CF chiave esterna not null, 
%           . MOTIVO not null
% . Specificare che la traduzione cattura la partecipazione (0,N)(1,1)
%   e non (1,N)(1,1), bisogna introdurre un vincolo esterno per garantire
%   (1,N)
% 
% - Traduzione di EFFETTATO-DURANTE 
%   . lo creiamo aggiungendo una chiave esterna su DIAGNOSI con not null
% 
% - Traduzione di CARTELLA-CLINICA
%   . lo creiamo aggiungendo una chiave esterna su DIAGNOSI con not null
% 
% - Traduzione di SOMMINISTRATO-DURANTE
%   . lo creiamo aggiungendo una chiave esterna su TERAPIA con not null
% 
% - Traduzione di CURATA-DA e di ISTANZA-DI e EFFETTO-COLLATERALE
%   . ISTANZA-DI-TERAPIA (__DIAGNOSI_,_TERAPIA__,...,DIAGNOSI-EC) 
%   DIAGNOSI e TERAPIA e DIAGNOSI-EC sono chiavi esterne.
%   La coppia DIAGNOSI e TERAPIA è chiave. 
%   Si deve inserire UNIQUE per DIAGNOSI e non per la coppia per 
%   garantire (1,1).      
%   Si deve inserire un vincolo esterno per garantire (1,N) su terapia
%   Si deve inserire UNIQUE su DIAGNOSI-EC per garantire 1 tra DIAG e EC




\clearpage
\section{Progettazione Fisica}
\subsection{Nuovi Indici}




\clearpage
\section{Definizione della Base di Dati in SQL}
\subsection{Definizione delle Tabelle}
\subsection{Popolamento della Base di Dati}
\subsection{Definizione di Query Significative}




\clearpage
\section{Analisi Dati in R}




\clearpage
\section{Portale Web}
\subsection{Interfaccia Grafica}
\subsection{Comunicare con pgAdmin}



\end{document}

\cleardoublepage
% Code example
\begin{lstlisting}[language=bash,title={bash version}]
#!/bin/bash
echo "Hello , world!"
\end{lstlisting}
